% !TeX TXS-program:compile = txs:///arara
% arara: pdflatex: {shell: yes, synctex: no, interaction: batchmode}
% arara: pdflatex: {shell: yes, synctex: no, interaction: batchmode} if found('log', '(undefined references|Please rerun|Rerun to get)')

\documentclass[french,a4paper,11pt]{article}
\usepackage[margin=2cm,includefoot]{geometry}
\def\TPversion{0.1.0}
\def\TPdate{02 Mai 2023}
\usepackage[utf8]{inputenc}
\usepackage[T1]{fontenc}
\usepackage{amsmath,amssymb}
\usepackage{tikz3d-fr}
\usepackage{awesomebox}
\usepackage{fontawesome5}
\usepackage{footnote}
\makesavenoteenv{tabular}
\usepackage{enumitem}
\usepackage{tabularray}
\usepackage{fancyvrb}
\usepackage{fancyhdr}
\fancyhf{}
\renewcommand{\headrulewidth}{0pt}
\lfoot{\sffamily\small [tikz3d-fr]}
\cfoot{\sffamily\small - \thepage{} -}
\rfoot{\hyperlink{matoc}{\small\faArrowAltCircleUp[regular]}}

%\usepackage{hvlogos}
\usepackage{hologo}
\providecommand\tikzlogo{Ti\textit{k}Z}
\providecommand\TeXLive{\TeX{}Live\xspace}
\providecommand\PSTricks{\textsf{PSTricks}\xspace}
\let\pstricks\PSTricks
\let\TikZ\tikzlogo
\newcommand\TableauDocumentation{%
	\begin{tblr}{width=\linewidth,colspec={X[c]X[c]X[c]X[c]X[c]X[c]},cells={font=\sffamily}}
		{\LARGE \LaTeX} & & & & &\\
		& {\LARGE \hologo{pdfLaTeX}} & & & & \\
		& & {\LARGE \hologo{LuaLaTeX}} & & & \\
		& & & {\LARGE \TikZ} & & \\
		& & & & {\LARGE \TeXLive} & \\
		& & & & & {\LARGE \hologo{MiKTeX}} \\
	\end{tblr}
}

\usepackage{hyperref}
\urlstyle{same}
\hypersetup{pdfborder=0 0 0}
\setlength{\parindent}{0pt}
\definecolor{LightGray}{gray}{0.9}

\usepackage{babel}
\AddThinSpaceBeforeFootnotes
\FrenchFootnotes

\usepackage[most]{tcolorbox}
\usetikzlibrary{calc}
\tcbuselibrary{minted}
\NewTCBListing{PresCodeTex}{ O{cyan} m }{%
	enhanced,width=0.93\linewidth,flush right,boxrule=0.75pt,colframe=#1!85!black,%
	sharp corners,top=0mm,bottom=0mm,left=0.4em,right=5mm,%
	before skip=\baselineskip,after skip=\baselineskip,%
	colback=white,
	fontupper=\footnotesize,fontlower=\footnotesize,%
	watermark text={\faCode},watermark opacity=0.25,watermark zoom=0.50,%
	title={{\scriptsize\faCode} Code \LaTeX},
	lefttitle=0.4em,
	fonttitle=\bfseries\footnotesize\sffamily,colbacktitle=darkgray!50!#1,%
	listing engine=minted,minted style=colorful,minted language=tex,
	minted options={tabsize=4,fontsize=\footnotesize,autogobble,breaklines=true},
	#2,%
	overlay={\draw[#1!85!black] ($(frame.north west)+(-0.035\linewidth,-0.025\linewidth)$) node[scale=1.66] {\faCode} ;}
}

\NewTCBListing{PresCode}{ O{cyan} m }{%
	enhanced,width=0.93\linewidth,flush right,boxrule=0.75pt,colframe=#1!85!black,%
	sharp corners,top=0mm,bottom=0mm,left=0.4em,right=5mm,%
	before skip=\baselineskip,after skip=\baselineskip,%
	colback=white,
	fontupper=\footnotesize,fontlower=\footnotesize,%
	watermark text={\faCogs},watermark opacity=0.25,watermark zoom=0.50,%
	title={{\scriptsize\faCogs} Code \LaTeX{} et sortie \LaTeX},
	lefttitle=0.4em,
	fonttitle=\bfseries\footnotesize\sffamily,colbacktitle=darkgray!50!#1,%
	listing engine=minted,minted style=colorful,minted language=tex,
	minted options={tabsize=4,fontsize=\footnotesize,autogobble,breaklines=true},
	#2,%
	overlay={%
		\draw[#1!85!black] ($(frame.north west)+(-0.035\linewidth,-0.025\linewidth)$) node[scale=1.66] {\faCode} ;
		\draw[#1!85!black] ($(segmentation.west)+(-0.035\linewidth,-0.025\linewidth)$) node[scale=1.66] {\faFilePdf} ;
		
	}
}

\NewTCBListing{PresCodeSortie}{ O{cyan} m }{%
	enhanced,width=0.93\linewidth,flush right,boxrule=0.75pt,colframe=#1!85!black,%
	sharp corners,top=0mm,bottom=0mm,left=0.4em,right=5mm,%
	before skip=\baselineskip,after skip=\baselineskip,%
	colback=white,
	fontupper=\footnotesize,fontlower=\footnotesize,%
	watermark text={\faFilePdf},watermark opacity=0.25,watermark zoom=0.50,%
	title={{\scriptsize\faFilePdf} Sortie \LaTeX},
	lefttitle=0.4em,
	fonttitle=\bfseries\footnotesize\sffamily,colbacktitle=darkgray!50!#1,%
	listing engine=minted,minted style=colorful,minted language=tex,
	minted options={tabsize=4,fontsize=\footnotesize,autogobble,breaklines=true},
	#2,%
	overlay={\draw[#1!85!black] ($(frame.north west)+(-0.035\linewidth,-0.025\linewidth)$) node[scale=1.66] {\faFilePdf} ;}
}

\NewTCBListing{PresentationCode}{ O{blue} m }{%
	sharp corners=downhill,enhanced,arc=12pt,skin=bicolor,%
	colback=#1!1!white,colframe=#1!75!black,colbacklower=white,%
	attach boxed title to top right={yshift=-\tcboxedtitleheight},title=Code \LaTeX,%
	boxed title style={%
		colframe=#1!75!black,colback=#1!15!white,%
		,sharp corners=downhill,arc=12pt,%
	},%
	top=\baselineskip,%
	fonttitle=\color{#1!90!black}\itshape\ttfamily\footnotesize,%
	listing engine=minted,minted style=colorful,
	minted language=tex,minted options={tabsize=4,fontsize=\small,autogobble},
	#2
}

\tcbset{vignettes/.style={%
	nobeforeafter,box align=base,boxsep=0pt,enhanced,sharp corners=all,rounded corners=southeast,%
	boxrule=0.75pt,left=7pt,right=1pt,top=0pt,bottom=0.25pt,%
	}
}

\tcbset{vignetteMaJ/.style={%
	fontupper={\vphantom{pf}\footnotesize\ttfamily},
	vignettes,colframe=ForestGreen!50!black,coltitle=white,colback=purple!25,%
	overlay={\begin{tcbclipinterior}%
			\fill[fill=purple!75]($(interior.south west)$) rectangle node[rotate=90]{\tiny \sffamily{\textcolor{Black}{\scalebox{0.85}[0.75]{\textbf{MàJ}}}}} ($(interior.north west)+(5pt,0pt)$);%
	\end{tcbclipinterior}}
	}
}

\newcommand\Cle[1]{{\bfseries\sffamily\textlangle #1\textrangle}}
\newcommand\cmaj[1]{\tcbox[vignetteMaJ]{#1}\xspace}

\begin{document}

\setlength{\aweboxleftmargin}{0.07\linewidth}
\setlength{\aweboxcontentwidth}{0.93\linewidth}
\setlength{\aweboxvskip}{8pt}

\pagestyle{fancy}

\thispagestyle{empty}

\vspace{2cm}

\begin{center}
	\begin{minipage}{0.75\linewidth}
	\begin{tcolorbox}[colframe=yellow,colback=yellow!15]
		\begin{center}
			\begin{tabular}{c}
				{\Huge \texttt{tikz3d-fr}}\\
				\\
				{\LARGE Quelques commandes (fr) pour} \\
				\\
				{\LARGE un peu de 3D avec \TikZ.} \\
			\end{tabular}
			
			\bigskip
			
			{\small \texttt{Version \TPversion{} -- \TPdate}}
		\end{center}
	\end{tcolorbox}
\end{minipage}
\end{center}

\begin{center}
	\begin{tabular}{c}
	\texttt{Cédric Pierquet}\\
	{\ttfamily c pierquet -- at -- outlook . fr}\\
	\texttt{\url{https://github.com/cpierquet/tikz3dfr}}
\end{tabular}
\end{center}

\vspace{0.25cm}

{$\blacktriangleright$~~Un environnement avec déclaration des unités.}

\smallskip

{$\blacktriangleright$~~Une commande pour afficher un pavé, avec personnalisations.}

\smallskip

{$\blacktriangleright$~~Deux commandes pour afficher des empilements de \og petits cubes \fg.}

\vspace{1cm}

\begin{center}
	\begin{tcolorbox}[enhanced,colframe=cyan,colback=cyan!2,center,width=0.95\linewidth,drop fuzzy shadow=lightgray]
	\begin{EnvTikzEspace}
		\PaveTikzTriDim[Cube,Largeur=2.5,AffLabel,Couleur=blue!50!black,CouleurSommets=red]
	\end{EnvTikzEspace}
	
	\vspace*{-2cm}
	
	\hfill\BlocPetitsCubes[violet]{1,2,3,4,5 / 0,2,1,1,3 / 0,2,1,1,2 / 1,1,1,1,1}\hfill~
	
	\vspace*{-2cm}
	
	\hfill
	\begin{EmpilementCubes}
		\PlaquePetitsCubes[cyan/red/green/yellow]{ 3111-1-4 / 233123 / 2-1112 }
		\PlaquePetitsCubes[cyan/red]{ 111 / -21 }
		\PlaquePetitsCubes[cyan/red/green]{ 3-2 }
	\end{EmpilementCubes}
	\end{tcolorbox}
\end{center}

\vspace{0.5cm}

%\hfill{}\textit{Merci à Denis Bitouzé et à Gilles Le Bourhis pour leurs retours et idées !}

\smallskip

\vfill

\hrule

\medskip

\TableauDocumentation

\medskip

\hrule

\medskip

\newpage

\phantomsection
\hypertarget{matoc}{}

\tableofcontents

\vfill

\part{Historique}

\verb|v0.1.0|~:~~~~Version initiale.

\newpage

\part{Introduction}

\section{Le package tikz3d-fr}

\subsection{Introduction}

\begin{noteblock}
Le package propose des commandes basiques -- francisées -- pour travailler sur des figures simples en 3D, à l'aide de \TikZ{} en utilisant des coordonnées tridimensionnelles :

\begin{itemize}
	\item un environnement avec gestion des unités $x/y/z$ ;
	\item une commande pour tracer et personnaliser un cube ;
	\item des commandes pour créer/afficher/nommer des points de l'espace ;
	\item des commandes pour tracer un ou plusieurs segments ;
	\item des commandes et un environnement pour travailler sur des \textit{empilements} de cubes.
\end{itemize}
\vspace*{-\baselineskip}\leavevmode
\end{noteblock}

\begin{importantblock}
Il existe d'autres solutions pour travailler avec de la 3D en \LaTeX, comme par exemple les packages \textsf{ProfCollege}\footnotemark\footnotetext{\url{https://www.ctan.org/pkg/profcollege}} (de Christophe Poulain, qui utilise \hologo{MetaPost}, et qui est certainement beaucoup plus performant) ou \textsf{pst-ob3d}\footnotemark\footnotetext{\url{https://www.ctan.org/pkg/pst-ob3d}} (de Herbert Voß et Denis Girou, qui utilise \pstricks).

\smallskip

L'idée est de proposer une utilisation des capacités (natives) 3D de \TikZ, en proposant des commandes \textit{simplifiées} et \textit{francisées} pour des figures simples (tétraèdres, cubes, pyramides, pavés) utilisées fréquemment dans des exercices de géométrie dans l'espace dans l'enseignement secondaires en France.
\end{importantblock}

\subsection{Chargement du package, packages utilisés}

\begin{importantblock}
Le package se charge, de manière classique, dans le préambule.

Il n'existe pas d'option pour le package, et \texttt{xcolor} n'est pas chargé avec des options spécifiques.
\end{importantblock}

\begin{PresCodeTex}{listing only}
\usepackage{tikz3d-fr}
\end{PresCodeTex}

\begin{noteblock}
\textsf{tikz3d-fr} charge les packages suivantes :

\begin{itemize}
	\item \texttt{tikz}, \texttt{xstring}, \texttt{xintexpr}, \texttt{simplekv} et \texttt{xinttools} et \texttt{listofitems} ;
	\item les librairies \texttt{\textit{tikz}.calc} et \texttt{\textit{tikz}.babel}.
\end{itemize}

Il est compatible avec les compilations usuelles en \textsf{latex}, \textsf{pdflatex}, \textsf{lualatex} ou \textsf{xelatex}.
\end{noteblock}

\vfill~

\pagebreak

\part{Environnement 3D et commandes de base}

\section{Création de l'environnement}

\subsection{Commande}

\begin{cautionblock}
L'environnement dédié à la création de figures en 3D avec \TikZ{} est \texttt{EnvTikzEspace}.

Il permet de définir les unités et \textit{angles} des différents axes.
\end{cautionblock}

\begin{PresCodeTex}{listing only}
\begin{EnvTikzEspace}[Clés]<options tikz>
	%commandes
\end{EnvTikzEspace}
\end{PresCodeTex}

\subsection{Clés et options}

\begin{tipblock}
Le premier argument, optionnel et entre \texttt{[...]} propose les \Cle{clés} suivantes :

\begin{itemize}
	\item \Cle{UniteX}, à donner sous la forme \Cle{\{angle\string:longueur\}} pour le vecteur de base $\vec{\imath}$ ;
	
	\hfill{}défaut : \Cle{\{-15\string:1cm\}},%
	\item \Cle{UniteY}, à donner sous la forme \Cle{\{angle\string:longueur\}} pour le vecteur de base $\vec{\jmath}$ ;
	
	\hfill{}défaut : \Cle{\{20\string:0.65cm\}}
	\item \Cle{UniteZ}, à donner sous la forme \Cle{\{angle\string:longueur\}} pour le vecteur de base $\vec{k}$ ;
	
	\hfill{}défaut : \Cle{\{90\string:1cm\}}
	\item Le booléen \Cle{VueClassique} pour un affichage en perspective habituelle.\hfill~défaut : \Cle{false}
\end{itemize}
\vspace*{-\baselineskip}\leavevmode
\end{tipblock}

\begin{tipblock}
Le second argument, optionnel et entre \texttt{<...>} est quant à lui relatif à des arguments à passer à l'environnement \TikZ{} créé, comme par exemple un alignement vertical, etc
\end{tipblock}

\begin{PresCode}{}
\begin{EnvTikzEspace}
	\draw[->,>=latex] (0,0,0)--(1,0,0) node[right] {$\vec{\imath}$} ;
	\draw[->,>=latex] (0,0,0)--(0,1,0) node[right] {$\vec{\jmath}$} ;
	\draw[->,>=latex] (0,0,0)--(0,0,1) node[above] {$\vec{k}$} ;
\end{EnvTikzEspace}
\end{PresCode}

\begin{PresCode}{}
\begin{EnvTikzEspace}[UniteX={-145:1.25cm},UniteY={0:1.75cm},UniteZ={90:1.75cm}]
	\draw[->,>=latex] (0,0,0)--(1,0,0) node[right] {$\vec{\imath}$} ;
	\draw[->,>=latex] (0,0,0)--(0,1,0) node[right] {$\vec{\jmath}$} ;
	\draw[->,>=latex] (0,0,0)--(0,0,1) node[above] {$\vec{k}$} ;
\end{EnvTikzEspace}
\end{PresCode}

\begin{PresCode}{}
\begin{EnvTikzEspace}[VueClassique]
	\draw[->,>=latex] (0,0,0)--(1,0,0) node[right] {$\vec{\imath}$} ;
	\draw[->,>=latex] (0,0,0)--(0,1,0) node[right] {$\vec{\jmath}$} ;
	\draw[->,>=latex] (0,0,0)--(0,0,1) node[above] {$\vec{k}$} ;
\end{EnvTikzEspace}
\end{PresCode}

\begin{noteblock}
Une fois l'environnement, via son \textit{repère}, est créé, toute commande en \TikZ{} est utilisable avec les coordonnées \texttt{(x,y,z)} ou bien les \texttt{scope} avec les \texttt{canva}.
\end{noteblock}

\begin{PresCode}{}
\begin{EnvTikzEspace}[UniteX={-145:0.575cm},UniteY={0:0.7cm},UniteZ={90:0.7cm}]
	\filldraw[red] (2,4,3) circle[radius=1pt] ; \draw[red] (0,0,0) -- (2,4,3) ;
	\draw[thin,->,>=latex] (0,0,0)--(3,0,0) ;
	\draw[thin,->,>=latex] (0,0,0)--(0,5,0) ;
	\draw[thin,->,>=latex] (0,0,0)--(0,0,4) ;
	\draw[->,>=latex] (0,0,0)--(1,0,0) ;
	\draw[->,>=latex] (0,0,0)--(0,1,0) ;
	\draw[->,>=latex] (0,0,0)--(0,0,1) ;
	\draw[dashed] (0,0,3)--(2,4,3)--(2,4,0)--(0,0,0) (2,4,0)--(2,0,0) (2,4,0)--(0,4,0) ;
\end{EnvTikzEspace}
\end{PresCode}

\pagebreak

\section{Points et segments}

\subsection{Commandes}

\begin{cautionblock}
Les commandes \textit{simplifiées} et \textit{francisées} disponible sont :

\begin{itemize}
	\item \texttt{\textbackslash PlacePointEspace} pour placer un point dans l'espace ;
	\item \texttt{\textbackslash PlacePointsEspace} pour placer des points dans l'espace ;
	\item \texttt{\textbackslash MarquePointEspace} pour marquer (matérialiser) un point dans l'espace ;
	\item \texttt{\textbackslash MarquePointsEspace} pour marquer (matérialiser) un point dans l'espace ;
	\item \texttt{\textbackslash TraceSegmentEspace} pour tracer un segment dans l'espace ;
	\item \texttt{\textbackslash TraceSegmentsEspace} pour tracer des segments dans l'espace.
\end{itemize}
\vspace*{-\baselineskip}\leavevmode
\end{cautionblock}

\begin{PresCodeTex}{listing only}
\begin{EnvTikzEspace}[Clés]<options tikz>
	%créer/placer/nommer un point
	\PlacePointEspace(*)[clés]{nœud}{coordonnées}<label>
	%créer/placer/nommer plusieurs points
	\PlacePointsEspace(*)[clés]{liste}
	%marquer un point
	\MarquePointEspace[clés]{point}
	%marquer plusieurs points
	\MarquePointsEspace[clés]{liste}
	%tracer un segment
	\TraceSegmentEspace[clés](point)(point)
	%tracer plusieurs segments
	\TraceSegmentsEspace[clés]{liste}
\end{EnvTikzEspace}
\end{PresCodeTex}

\subsection{Clés, options et arguments}

\begin{tipblock}
Les versions étoilées désactivent l'affichage des labels des points.

\smallskip

L'argument optionnel et entre \texttt{[...]} propose les \Cle{clés} suivantes (communes ou spécifiques) :

\begin{itemize}
	\item \Cle{PosLabel} pour préciser la position (francisée) du label pour les points ; \hfill{}défaut : \Cle{b}
	\item \Cle{StyleMarque} parmi \Cle{x/o} pour spécifier le style de la marque des points ;
	
	\hfill{}défaut : \Cle{o}
	\item \Cle{TailleMarque} pour spécifier la taille de la marque des points (disque ou croix) ;
	
	\hfill{}défaut : \Cle{2pt}
	\item \Cle{Couleur} pour paramétrer la couleur.\hfill{}défaut : \Cle{black}
\end{itemize}
\vspace*{-\baselineskip}\leavevmode
\end{tipblock}

\begin{tipblock}
Les positions pour les labels des points sont \textit{francisées} :

\begin{itemize}
	\item \Cle{b} : \texttt{bas}
	\item \Cle{h} : \texttt{haut}
	\item \Cle{g} : \texttt{gauche}
	\item \Cle{d} : \texttt{droite}
	\item \Cle{hg} : \texttt{haut gauche}
	\item \ldots
\end{itemize}
\vspace*{-\baselineskip}\leavevmode
\end{tipblock}

\begin{tipblock}
De manière un peu plus spécifique :

\begin{itemize}
	\item le \texttt{<label>} pour la commande \texttt{\textbackslash PlacePointEspace} est optionnel, et est identique à \texttt{\{nœud\}} ;
	\item la \texttt{\{liste\}} pour la commande \texttt{\textbackslash PlacePointsEspace} est à donner -- par exemple -- sous la forme \texttt{A/0,0,0/bg B/5,2,1/hd} ;
	\item la \texttt{\{liste\}} pour la commande \texttt{\textbackslash TraceSegmentsEspace} est à donner -- par exemple -- sous la forme \texttt{A/B A/C A/D B/D} ;
	\item les \texttt{[clés]} pour les segments correspondent aux options en langage \TikZ.
\end{itemize}
\vspace*{-\baselineskip}\leavevmode
\end{tipblock}

\begin{PresCode}{}
\begin{EnvTikzEspace}[VueClassique]
	%placement des points avec labels
	\PlacePointsEspace[Couleur=red]{A/0,0,0/bg B/4,0,0/bd C/4,4,0/hd D/0,4,0/hg E/0,0,4/bg F/4,0,4/bd G/4,4,4/hd H/0,4,4/hg}
	%segments pointillés
	\TraceSegmentsEspace[thick,dashed,purple]{A/D D/C D/H}
	%segments pleins
	\TraceSegmentsEspace[thick,blue]{A/B B/C C/G G/H H/E E/A E/F B/F F/G}
	%Marques points
	\MarquePointsEspace[Couleur=orange]{A,B,C,D,E,F,G,H}
\end{EnvTikzEspace}
\end{PresCode}

\pagebreak

\part{Empilements de cubes}

\section{Environnement dédié}

\begin{cautionblock}
L'environnement dédié à la création de figures type \textit{empilement de cubes} est\ldots{} \texttt{EmpilementCubes}.

Il existe deux manières de définir les empilements :

\begin{itemize}
	\item en travaillant par \textit{plaques} verticales de l'arrière vers l'avant ;
	\item en travaillant par les \textit{hauteurs} des colonnes, de l'arrière vers l'avant \textbf{et} de gauche à droite.
\end{itemize}

L'unité de base des cubes est fixée au départ à $0{,}5$\,cm.
\end{cautionblock}

\begin{importantblock}
Les axes (et de ce fait la vue proposée !) sont fixés, non modifiables, donc cette partie est beaucoup moins performante que ce propose le package \textsf{ProfCollege} avec sa commande \texttt{\textbackslash VueCubes} !
\end{importantblock}

\begin{PresCodeTex}{listing only}
\begin{EmpilementCubes}[échelle]<options tikz>
	%commandes
\end{EmpilementCubes}
\end{PresCodeTex}

\section{Création par \textit{plaques}}

\subsection{Commandes}

\begin{cautionblock}
La commande pour créer une plaque \textit{verticale} est \texttt{\textbackslash PlaquePetitsCubes}, avec la contrainte de créer la \textit{figure} de l'arrière vers l'avant.

Il existe également la commande \texttt{\textbackslash PlaqueVide} pour \textit{passer} une ligne.
\end{cautionblock}

\begin{PresCodeTex}{listing only}
\begin{EmpilementCubes}[échelle]<options tikz>
	\PlaquePetitsCubes[couleur(s)]{empilement}
	\PlaqueVide[nb]
\end{EmpilementCubes}
\end{PresCodeTex}

\begin{importantblock}
Les plaques créées sont affichées l'une \textit{devant} l'autre, et elles sont -- par défaut -- collées les unes aux autres.
\end{importantblock}

\subsection{Options et arguments}

\begin{tipblock}
Le premier argument, optionnel et entre \texttt{[...]} permet de spécifier une couleur (\texttt{cyan} par défaut) ou une liste de couleurs qui seront utilisées pour la création des plaques :

\begin{itemize}
	\item soit une couleur unique, qui sera \textit{codée} par \texttt{1} pour la création des cubes ;
	\item soit plusieurs couleurs, sous la forme \texttt{couleur1/couleur2/couleur3/...} qui seront codées par \texttt{1}, \texttt{2}, \ldots{} pour la création des cubes.
\end{itemize}

Le second argument, obligatoire et entre \texttt{\{...\}} est quant à lui la liste, des lignes à construire, avec comme ordres :

\begin{itemize}
	\item du bas vers le haut (caractère de séparation \texttt{/}) ;
	\item de la gauche vers la droite (caractère de séparation \texttt{,}) ;
	\item un \texttt{-} code un \textit{trou}, et un numéro code une couleur (comme définie(s) précédemment).
\end{itemize}

Cette manière permet de créer des plaques avec couleurs \textit{individuelles} et des \textit{trous} éventuels.
\end{tipblock}

\begin{PresCode}{}
\begin{EmpilementCubes}
	%plaque
	%de bas en haut : couleurs VBRB-B-J / RVVBRV / J-BBBR
	\PlaquePetitsCubes[cyan/red/green/yellow]{ 3121-1-4 / 233123 / 4-1112 }
\end{EmpilementCubes}
\end{PresCode}

\begin{PresCode}{}
\begin{EmpilementCubes}
	%plaque n°1 (fond)
	\PlaquePetitsCubes{ 1111 / 1111 }
	\PlaqueVide[2]
	%plaque n°2
	\PlaquePetitsCubes[cyan/red]{ 111- / -211  }
	\PlaqueVide[2]
	%plaque n°3 (devant)
	\PlaquePetitsCubes{ 1111 / ---1 }
\end{EmpilementCubes}
\end{PresCode}

\begin{PresCode}{}
\begin{EmpilementCubes}
	%plaque n°1 (fond)
	\PlaquePetitsCubes{ 1111 / 1111 }
	%plaque n°2
	\PlaquePetitsCubes[cyan/red]{ 111- / -211  }
	%plaque n°3 (devant)
	\PlaquePetitsCubes{ 1111 / ---1 }
\end{EmpilementCubes}
\end{PresCode}


\begin{PresCode}{}
\begin{EmpilementCubes}[1.75]
	\PlaquePetitsCubes[cyan/red/green/yellow]{ 3111-1-4 / 233123 / 2-1112 }
	\PlaquePetitsCubes[cyan/red]{ 111 / -21 }
	\PlaquePetitsCubes[cyan/red/green]{ 3-2 }
\end{EmpilementCubes}
\end{PresCode}

\pagebreak

\section{Création par \textit{hauteurs}}

\subsection{Commande}

\begin{tipblock}
L'idée, reprise du package \textsf{ProfCollege}\footnotemark\footnotetext{\url{https://www.ctan.org/pkg/profcollege}} permet d'afficher une empilement de cubes (monochromes, et sans trou) en précisant -- grâce à un système de \textit{grille} -- les hauteurs des colonnes.

La commande qui permet de réaliser cet empilement est \texttt{\textbackslash BlocPetitsCubes}.
\end{tipblock}

\begin{PresCodeTex}{listing only}
%création dans un environnement dédié
\begin{EmpilementCubes}[échelle]<options tikz>
	\BlocPetitsCubes*[couleur]{grille des hauteurs}
\end{EmpilementCubes}
\end{PresCodeTex}

\begin{PresCodeTex}{listing only}
%création autonome
\BlocPetitsCubes[couleur]{grille des hauteurs}
\end{PresCodeTex}

\begin{cautionblock}
Le support de la grille des hauteurs est donc à donner sous forme \textit{rectangulaire}, en respectant un nombre \textit{homogène} de colonnes par ligne !
\end{cautionblock}

\begin{importantblock}
Voici une grille permettant d’anticiper la création d'un assemblage (en spécifiant les hauteurs) :

\begin{center}
	\begin{EnvTikzEspace}[UniteX={-22:1cm},UniteY={22:1cm}]
		\foreach \x in {0,...,4} {\draw[semithick] (\x,-0.2,0)--++(0,5.4,0) ;}
		\foreach \y in {0,...,5} {\draw[semithick] (-0.2,\y,0)--++(4.4,0,0) ;}
		\foreach \y/\n in {0.5/1,1.5/2,2.5/3,3.5/4,4.5/5}
			{\draw (0.5,\y,0) node[font=\small\sffamily] {\n} ;}
		\foreach \y/\n in {0.5/0,1.5/2,2.5/1,3.5/1,4.5/3}
			{\draw (1.5,\y,0) node[font=\small\sffamily] {\n} ;}
		\foreach \y/\n in {0.5/0,1.5/2,2.5/1,3.5/1,4.5/2}
			{\draw (2.5,\y,0) node[font=\small\sffamily] {\n} ;}
		\foreach \y/\n in {0.5/1,1.5/1,2.5/1,3.5/1,4.5/1}
			{\draw (3.5,\y,0) node[font=\small\sffamily] {\n} ;}
	\end{EnvTikzEspace}
\end{center}
\end{importantblock}

\subsection{Options et arguments}

\begin{tipblock}
Le premier argument, optionnel et entre \texttt{[...]} permet de spécifier une couleur (\texttt{cyan} par défaut).

Le second argument, obligatoire et entre \texttt{\{...\}} est quant à lui la liste des hauteurs, comme présentée précédemment :

\begin{itemize}
	\item le caractère de séparation des \textit{plaques} est le \texttt{/} ;
	\item pour chaque plaque, le caractère de séparation des colonnes est le \texttt{,}.
\end{itemize}
\vspace*{-\baselineskip}\leavevmode
\end{tipblock}

\begin{importantblock}
Par exemple, la \textit{grille} associée à l'empilement précédent est :

\smallskip

\hfill\texttt{1,2,3,4,5 / 0,2,1,1,3 / 0,2,1,1,2 / 1,1,1,1,1}\hfill~
\end{importantblock}

\vfill~

\begin{PresCode}{}
\begin{EmpilementCubes}
	\BlocPetitsCubes*{1,2,3,4,5}
\end{EmpilementCubes}
\end{PresCode}

\begin{PresCode}{}
%commande autonome, taille par défaut
\BlocPetitsCubes{1,2,3,4,5 / 0,2,1,1,3 / 0,2,1,1,2 / 1,1,1,1,1}
\end{PresCode}

\begin{PresCode}{}
\begin{EmpilementCubes}[2]
	\BlocPetitsCubes*[violet]{1,2,3,4,5 / 0,2,1,1,3 / 0,2,1,1,2 / 1,1,1,1,1}
\end{EmpilementCubes}
\end{PresCode}

\end{document}